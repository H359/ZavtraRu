\chapter{Рубрикация и содержание}
    \section{Общее устройство сайта}
        Сайт газеты <<Завтра>> по существу представляет собой набор <<единиц контента>>, объединенных рубриками (в т.ч. <<пустой>> рубрикой),
        каждая единица контента, рубрика или любой другой объект может быть снабжён возможностью комментирования, а также тегами.
        
        Управление сайтом производится через административный интерфейс, находящийся по адресу \verb+сайт_газеты/admin+. Вход в административный
        интерфейс осуществляется по паре логин-пароль, каждый пользователь, имеющий туда доступ наделён набором полномочий --- возможностью управлять
        только объектами, подконтрольными ему в силу должностных обязанностей.
        
        Все сущности сайта сгруппированы по <<приложениям>>, в зависимости от их предназнаения. На главной странице административного интерфейсе сайта
        представлен набор <<приложений>>, в каждом из которых находится список доступных данному пользователю сущностей и действий с ними.
    \section{Рубрика}
        Рубрики --- это именованная группа единиц контента. Она имеет следующий набор характеристик и полей:
        \begin{itemize}
            \item Название --- наименование рубрики, под этим названием она выступает во всех публичных частях сайта;
            \item Выводить на главной --- флаг, определеяющий попадание материалов рубрики в основную <<ленту>> на главной странице сайта;
            \item Выводить в верхнем большом меню --- флаг, определяющий попадание рубрики в виде ссылки в верхнее меню сайта;
            \item Позиция --- число, устанавливающее порядок на множестве рубрик;
        \end{itemize}
        Управление рубриками осуществляется в разделе <<Corecontent>> в пункте <<Рубрики>> административного интерфейса.
    \section{Горячая тема}
        Горячая тема --- набор единиц контента, отобранных по тегам (в т.ч. и по одному тегу). Горячие темы предназначены для группировки единиц контента
        в соответствии с каким-либо признаком и автоматической публикации их в специальных разделах сайта, что позволит более гибко освещать какие-либо
        события, не прибегая к созданию специальной рубрики и переназначению единиц контента на эту рубрику.

        Горячая тема может быть в любой момент снята с публикации на главной странице, путём изменения соответствующей характеристики. Это не означает
        исчезновение темы с сайта вообще, она всё также будет доступна по тем же адресам, однако на неё не будет указано ссылки в шапке сайта.

        Имеет следующий набор характеристик и полей:
        \begin{itemize}
            \item Название
            \item Выводить на главной / актуально --- флаг, определяющий отображение горячей темы в шапке сайта
            \item Теги --- набор тегов, по которым проводится отбор единиц контента в данную тему. Отбор происходит по предикату "ИЛИ", т.е. все материалы имеющие тег1 или тег2 или тег3 и т.д.
        \end{itemize}
    \section{Единица контента}
        Единица контента --- основная информационная сущность сайта. Все другие сущности --- статьи, видео-материалы, изображения, новости (и те, которые
        могут появиться в результате развития сайта) --- являются <<наследниками>> данной, то есть имеют те же поля и характеристики, интерпретируемые в
        зависимости от типа конкретизации. в том числе и скрываемые (так, для новостей нет смысла управлять рубрикой, т.к. для них определена единственная
        возможная рубрика --- <<Новости>>).
        
        Каждая информационная сущность-наследник сущности <<Единица контента>> сама является сущностью <<Единица контента>>. Однако управление именно 
        <<сырыми>> сущностями <<Единица контента>> не предусмотрено, т.к. не имеет смысла.
        
        Набор характеристик и полей:
        \begin{itemize}
            \item Заголовок
            \item Подзаголовок
            \item Рубрика --- отнесение единицы контента к рубрике, может быть пустым
            \item Анонс --- краткое описание единицы контента
            \item Дата публикации
            \item Авторы --- набор авторов публикации, может быть пустым
            \item Опубликовано в газете --- флаг, указывающий на то, был ли данный материал опубликован в каком-либо выпуске в бумажной газете
            \item Эскиз / маленькое изображение --- ссылка на т.н. тамбнейл материала, имеет смысл только для видео и изображений
            \item Содержимое
            \item URL на старом сайте --- используется для установления преемственности со старым сайтом газеты
            \item Теги --- набор тегов, описывающих данный материал
        \end{itemize}
        
        Единица контента, не отнесённая к какой-либо рубрике, попадает в специальную рубрику <<Блоги>>.
        
        Поле <<URL на старом сайте>> имеет смысл заполнять только в случае переноса материала со старого сайта. В этом случае в нём необходимо указать
        полный URL материала со старого сайта.
